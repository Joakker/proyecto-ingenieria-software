\documentclass{article}
\usepackage{graphicx} % Required for inserting images

\title{Ingeniería Software - Proyecto semestral \\ Entrega 1 }
\author{Grupo 5}
\date{September 2023}

\begin{document}

\maketitle

\section{Introducción}

Nuestra solución consiste en una plataforma virtual, la cual puede ser ingresada tanto por los médicos como por el personal administrativo, a fin de coordinar de mejor manera la atención de los pacientes a través del tiempo.

\section{Modelado de negocio}
\subsection*{Casos de usuario}
Como médico debo:
\begin{enumerate}
    \item{Tener acceso a la ficha clínica del paciente, pero sólo si está asignado a mí}
    \item{Poder coordinarme con los demás doctores que tratan a ese mismo paciente}
    \item{Saber a quién y cuándo debo tratar}
    \item{Dejar constancia de qué medicamentos debe tomar el paciente y cuáles no debe tomar}
\end{enumerate}
\vspace{5mm}
Como farmacéutico debo:
\begin{enumerate}
    \item{Saber qué medicamentos debe tomar cada paciente}
    \item{Saber qué medicamentos están disponibles}
\end{enumerate}
\vspace{5mm}
Como personal administrativo debo:
\begin{enumerate}
    \item{Poder agendar, mover y cancelar las horas médicas según sea necesario}
    \item{Revisar qué doctores están disponibles en un día}
\end{enumerate}

\section{Requerimientos}
\begin{enumerate}
    \item{Nombre, rut de los pacientes}
    \item{Existencias de medicamentos en la farmacia}
    \item{Nombre, especialidad, horarios de cada profesional que trabaje en el hospital}
\end{enumerate}
\end{document}

